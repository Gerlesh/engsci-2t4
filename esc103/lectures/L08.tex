\section{Lecture Eight: Matrix Multiplication}
\begin{itemize}
    \item When multiplying two matrices, the entry in row $i$ and column $j$ of $AB$ is:
    \begin{equation}
        \text{(Row i of A)} \cdot \text{(column j of B)}
    \end{equation}
    \item Recall that $A$ and $B$ can only be multiplied of $A$ is $m\times n$ and $B$ is $n\times p$. The size of the resulting matrix is therefore $m\times p$.
    \begin{example}
        Suppose we wish to multiply $A=\begin{bmatrix}
            2&4&8\\1&3&6
        \end{bmatrix}$ and $B=\begin{bmatrix}
            1&9 \\ 4&2 \\ 3&7
        \end{bmatrix}$. To determine $AB$, we get:
        \begin{equation}
            AB = \begin{bmatrix}
                2\cdot 1+4\cdot 4 + 8\cdot 3 & 2\cdot 9 + 4\cdot 2 + 8\cdot 7 \\ 
                1\cdot 1 + 3\cdot 4 + 3\cdot 6 & 1\cdot 9 + 3\cdot 2 + 6\cdot 7
            \end{bmatrix} = \begin{bmatrix}
                42 & 82 \\ 31 & 57
            \end{bmatrix}
            \label{eq:}
        \end{equation}
    \end{example}
    \item This leads properties of matrices:
    \begin{enumerate}
        \item $A+B=B+A$ (commutative)
        \item $c(A+B)=cA+cB$ (where $c$ is scalar)
        \item $A+(B+C)=(A+B)+C$ (associative)
        \item $C(A+B)=CA+CB$ (distributive from left)
        \item $(A+B)C=AC+BC$ (distributive from right)
        \item $A(BC)=(AB)C$ (associative)
    \end{enumerate}
    \item We can take exponents:
    \begin{align}
        AA&=A^2\\ 
        A^pA^q &= A^{p+q} \\ 
        \left(A^{p}\right)^q &= A^{pq}
        \label{eq:}
    \end{align}
    and later on we will see the inverse:
    \begin{equation}
        A^{-1}
        \label{eq:}
    \end{equation}
    \item We can also view matrices as \textbf{transformations}. A linear transformation $L$ is a function that maps that maps a vector in $\mathbb{R}^n$ to a vector in $\mathbb{R}^n$ with the following properties:
    \begin{equation}
        L:\mathbb{R}^n \to \mathbb{R}^n
        \label{eq:}
    \end{equation}
    This is analogous to the mapping:
    \begin{equation}
        y=f(x)
        \label{eq:}
    \end{equation}
    or
    \begin{equation}
        f:x\to y
        \label{eq:}
    \end{equation}
    \item If $\vec{v}$ and $\vec{w}$ $\in$ $\mathbb{R}^n$, then $L(\vec{v})$ and $L(\vec{w})$ $\in$ $\mathbb{R}^n$. It has the following properties:
    \begin{enumerate}
        \item $L(c\vec{v})=cL(\vec{v})$
        \item $L(\vec{v}+\vec{w})=L(\vec{v})+L(\vec{w})$
    \end{enumerate}
    \begin{example}
        Suppose we define a transformation $T_1$ that adds a constant vector $\vec{u}_0$ to every vector where:
        \begin{equation}
            T_1: \mathbb{R}^n \to \mathbb{R}^n
            \label{eq:}
        \end{equation}
        Is this a linear transformation? If so, then the following must be true:
        \begin{align}
            T_1(\vec{v})+T_2(\vec{w})&=T_1(\vec{v}+\vec{w}) \\ 
            \vec{v}+\vec{w}+2\vec{u}_0&=\vec{v}+\vec{w}+\vec{u}_0
        \end{align}
        which is only true when $\vec{u}$ is the zero vector.
    \end{example}
\end{itemize}