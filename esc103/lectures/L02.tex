\section{Lecture 2}
\begin{itemize}
    \item The heart of linear algebra are \textbf{linear combinations}. Let $c$ and $d$ be scalars. Then:
    \begin{equation}
        c\vec{v} + d\vec{w}
        \label{eq:}
    \end{equation}
    is a linear combination (LC) of $\vec{v}$ and $\vec{w}$.
    \item For vectors in three-dimensional ($\mathbb{R}^3$)
    \begin{center}
        \tdplotsetmaincoords{60}{120} 
\begin{tikzpicture} [scale=3, tdplot_main_coords, axis/.style={->,blue,thick}, 
vector/.style={-stealth,black,very thick}, 
vector guide/.style={dashed,red,thick}]

%standard tikz coordinate definition using x, y, z coords
\coordinate (O) at (0,0,0);

%tikz-3dplot coordinate definition using x, y, z coords

\pgfmathsetmacro{\ax}{0.8}
\pgfmathsetmacro{\ay}{0.8}
\pgfmathsetmacro{\az}{0.8}

\coordinate (P) at (\ax,\ay,\az);

%draw axes
\draw[axis] (0,0,0) -- (1,0,0) node[anchor=north east]{$x$};
\draw[axis] (0,0,0) -- (0,1,0) node[anchor=north west]{$y$};
\draw[axis] (0,0,0) -- (0,0,1) node[anchor=south]{$z$};

%draw a vector from O to P
\draw[vector] (O) -- (P) node[right] {$v=\begin{bmatrix}
    v_1\\v_2\\v_3
\end{bmatrix}$};

%draw guide lines to components
\draw[vector guide]         (O) -- (\ax,\ay,0);
\draw[vector guide] (\ax,\ay,0) -- (P);
\draw[vector guide]         (P) -- (0,0,\az);
\draw[vector guide] (\ax,\ay,0) -- (0,\ay,0);
\draw[vector guide] (\ax,\ay,0) -- (0,\ay,0);
\draw[vector guide] (\ax,\ay,0) -- (\ax,0,0);
\end{tikzpicture}
    % \begin{tikzpicture}[x=1cm, y=1cm, z=-0.6cm]
    %     % Axes
    %     \draw [->] (0,0,0) -- (4,0,0) node [right] {$x$};
    %     \draw [->] (0,0,0) -- (0,4,0) node [left] {$y$};
    %     \draw [->] (0,0,0) -- (0,0,4) node [left] {$z$};
    %     % Vectors
    %     \draw [->, thick] (0,0,0) -- (2,2,1);
    %     % Ticks
    %     \foreach \i in {1,2}
    %     {
    %     \draw (-0.1,\i,0) -- ++ (0.2,0,0);
    %     \draw (\i,-0.1,0) -- ++ (0,0.2,0);
    %     \draw (-0.1,0,\i) -- ++ (0.2,0,0);
    %     }
    %     % Dashed lines
    %     \draw [loosely dashed]
    %         (0,0,0) -- (2,0,0)
    %         (2,0,0) -- (2,2,0)
    %         (2,2,0) -- (2,2,1)
    %         ;
    %     % Labels
    %      \node [right] at (2,2,1) {$\vec{v}=\begin{bmatrix}
    %                                 2\\2\\1
    %                                \end{bmatrix}$};
    % \end{tikzpicture}
    \end{center}
    \item The \textbf{transpose} of a matrix changes swaps the $ig$ index with the $ji$ index such that:
    \begin{equation}
        \begin{bmatrix}
            v_1&v_2&v_3
        \end{bmatrix}^T = \begin{bmatrix}
            v_1\\v_2\\v_3
        \end{bmatrix}
        \label{eq:}
    \end{equation}
    \item Addition, subtraction, and multiplication with scalars behave in the same ways.
    \item There are a few properties that vectors behave:
        \begin{itemize}
            \item The \textbf{commutative property} says that:
            \begin{equation}
                \vec{v}+\vec{w}+\vec{w+\vec{v}}
                \label{eq:}
            \end{equation}
            \item The \textbf{associative property} says that;
            \begin{equation}
                \vec{v}+\vec{w}+\vec{z}=(\vec{v}+\vec{w})+\vec{z}=\vec{v}+(\vec{w}+\vec{z})
                \label{eq:}
            \end{equation}
            
        \end{itemize}
    \item To introduce \textbf{vector spaces}, suppose we have three vectors $\vec{v}$, $\vec{w}$, $\vec{z}$ and three scalars $c$, $d$, $e$.
    \begin{itemize}
        \item Linear combinations of just $c\vec{v}$ gives a one-dimensional line given that $c\neq 0$.
        \begin{center}
            \begin{tikzpicture}[x=1cm, y=1cm]
                Axes
                \draw [<->] (-3,0) -- (3,0) node [at end, right] {$x$};
                \draw [<->] (0,-3) -- (0,3) node [at end, left] {$y$};
                % node[below] {$O$}
    
                % Vectors
                \foreach \x in {-3,-2.5,...,3}
                \draw [->, thick] (0,0) -- (\x,0.5*\x);
            \end{tikzpicture}
        \end{center}
        \item Linear combination of $c\vec{v}+d\vec{w}$ given that $\vec{v}$ and $vec{w}$ are not parallel (not colinear) give a plane:
        \begin{center}
            \tdplotsetmaincoords{60}{120} 
    \begin{tikzpicture} [scale=4, tdplot_main_coords, axis/.style={->,blue,very thick}, 
    vector/.style={-stealth,black}]
    
    %standard tikz coordinate definition using x, y, z coords
    \coordinate (O) at (0,0,0);
    
    %tikz-3dplot coordinate definition using x, y, z coords
    \foreach \x in {-0.4,-0.3,...,1.2}
        \foreach \y in {-0.4,-0.3,...,1.2}{
            \pgfmathsetmacro{\ax}{\x}
            \pgfmathsetmacro{\ay}{\y}
            \pgfmathsetmacro{\az}{0.7}
            \coordinate (P) at (\ax,\ay,\az);
            \draw[vector] (O) -- (P);
        }
    


    %draw axes
    \draw[axis] (0,0,0) -- (1,0,0) node[anchor=north east]{$x$};
    \draw[axis] (0,0,0) -- (0,1,0) node[anchor=north west]{$y$};
    \draw[axis] (0,0,0) -- (0,0,1) node[anchor=south]{$z$};
    
    %draw a vector from O to P
    
    \end{tikzpicture}
    \end{center}
    \item The linear combinations of $c\vec{v}$+$d\vec{w}$+$e\vec{z}$ where the vectors are not coplanar (lie on the same plane) give all of $\mathbb{R}^3$.
    \begin{center}
        \tdplotsetmaincoords{60}{120} 
\begin{tikzpicture} [scale=4, tdplot_main_coords, axis/.style={->,blue,very thick}, 
vector/.style={-stealth,black}]

%standard tikz coordinate definition using x, y, z coords
\coordinate (O) at (0,0,0);

%tikz-3dplot coordinate definition using x, y, z coords
\foreach \x in {0,0.1,...,1}
    \foreach \y in {0,0.1,...,1}
    \foreach \z in {0,0.1,...,1}{
        \coordinate (P) at (\x,\y,\z);
        \draw[vector] (O) -- (P);
    }



%draw axes
\draw[axis] (0,0,0) -- (1,0,0) node[anchor=north east]{$x$};
\draw[axis] (0,0,0) -- (0,1,0) node[anchor=north west]{$y$};
\draw[axis] (0,0,0) -- (0,0,1) node[anchor=south]{$z$};

\end{tikzpicture}
\end{center}
\end{itemize}
\item The length of a vector of $\vec{v}$ in $\mathbb{R}^N$ is given by:
\begin{equation}
    \lVert v \rVert = \sqrt{v_1^2+v_2^2+\cdots+v_N^2}
    \label{eq:}
\end{equation}

\begin{prooof}
    We prove via induction. Suppose:
    \begin{equation}
        \lVert v \rVert = \sqrt{v_1^2+v_2^2+\cdots+v_N^2}
        \label{eq:}
    \end{equation}
    is true. We now prove that this is true for $\lVert w \rVert$ as well where $w$ is in $\mathbb{R}^{N+1}$. We can write:
    \begin{equation}
        \vec{v'}=\begin{bmatrix}
            v_1\\v_2\\v_3\\ \ddots \\ v_N \\ 0
        \end{bmatrix} ,\, 
        \vec{v''}=\begin{bmatrix}
            v_1\\v_2\\ v_3 \\ \ddots \\ v_{N-1} \\ v_N
        \end{bmatrix}
        ,\, \vec{v'''}=\begin{bmatrix}
            0\\0\ 0 \\ \ddots \\ 0 \\ v_N
        \end{bmatrix}
        \label{eq:}
    \end{equation}
    Since these two are orthogonal, we can use Pythagorean theorem:
    \begin{equation}
        \lVert\vec{v''}\rVert=\sqrt{(\lVert\vec{v'}\rVert)^2+(\lVert\vec{v'''}\rVert)^2}
        =\sqrt{v_1^2+v_2^2+\cdots+v_{N-1}^2}
        \label{eq:}
    \end{equation}
    Since this is true for $N=2$, this must be true for all $N$.
\end{prooof}
\item A few properties of the absolute magnitude:
    \begin{itemize}
        \item When multiplied by a scalar:
        \begin{equation}
            \lVert c\vec{v} \rVert = c\lvert \vec{v}\rvert
            \label{eq:}
        \end{equation}
        \item When the magnitude is zero:
        \begin{equation}
            \lVert \vec{v} \rVert = 0 \iff \vec{v}=\vec{0}
            \label{eq:}
        \end{equation}
    \end{itemize}
\item A \textbf{unit vector} is any vector with length equal to one. The most famous ones:
    \begin{equation}
        \vec{i}= \begin{bmatrix}
            1\\0\\0
        \end{bmatrix} ,\,
        \vec{j}=\begin{bmatrix}
            0\\1\\0
        \end{bmatrix},\,
        \vec{k}=\begin{bmatrix}
            0\\0\\1
        \end{bmatrix}
    \end{equation}
    \item To turn any vector $\vec{v}$ into a unit vector, we can multiply it by the scalar $\frac{1}{\lVert \vec{v} \rVert}$.
    \item Suppose we wish to find the distance between $P_1$ and $P_2$, we can use the following steps:
    \begin{enumerate}
        \item Define a coordinate system and orient the vectors in the given system. Draw a diagram.
        \item We can write the linear combination algebraically:
        \begin{equation}
            \overrightarrow{OP_1}+\overrightarrow{P_1P_2} = \overrightarrow{OP_2} \implies \overrightarrow{P_1P_2}=\overrightarrow{OP_2}-\overrightarrow{OP_1}
            \label{eq:}
        \end{equation}
        \item We can then solve for $\lVert\overrightarrow{P_1P_2}\rVert$ and we are done.
    \end{enumerate}
\end{itemize}