\section{Lecture 4: Projections}
\begin{itemize}
    \item For two nonzero vectors $\vec{v}$ and $\vec{w}$ such that:
    \begin{equation}
        \vec{v}\cdot\vec{w}=0\implies \cos\theta=0\implies \theta=\frac{\pi}{2}
        \label{eq:}
    \end{equation}
    If $\vec{v}$ and/or $\vec{w}$ is the zero vector,
    \begin{equation}
        \vec{v}\cdot\vec{w}=0
        \label{eq:}
    \end{equation}
    is also true.
    \begin{definition}
        $\vec{v}$ and $\vec{w}$ are orthogonal if and only if $\vec{v}\cdot\vec{w}=0$.
    \end{definition}
    \item Projections are what we use to define points in 2-d and 3-d. For example:
    \begin{center}
        \begin{tikzpicture}
            \draw[->] (0,0) -- (5,0) node[right] {$x$};
            \draw[->] (0,0) -- (0,5) node[right] {$y$};

            \draw[thick,->] (0,0) -- (3,4) node[right] {$(v_1,v_2)$};
            \draw[dotted] (0,4) node[left] {$(0,v_2)$} -- (3,4);
            \draw[dotted] (3,0) node[below] {$(v_1,0)$} -- (3,4);

            \draw[very thick,->] (0,0) -- (1,0) node[midway,below] {$\vec{i}=\begin{bmatrix}
                1\\0
            \end{bmatrix}$};
            \draw[very thick,->] (0,0) -- (0,1) node[midway,left] {$\vec{j}=\begin{bmatrix}
                1\\0
            \end{bmatrix}$};

            \draw[thick,->] (0,0) -- (0,3);
            \draw[thick,->] (0,0) -- (4,0);
        \end{tikzpicture}
    \end{center}
    We can write $\vec{v}=\begin{bmatrix}
        v_1\\v_2
    \end{bmatrix}$ as:
    \begin{align}
        \vec{v} &= v_1\vec{i}+v_2\vec{j} \\ 
        &= v_1\begin{bmatrix}
            1\\0
        \end{bmatrix} + v_2\begin{bmatrix}
            0\\ 1
        \end{bmatrix}
        \\ 
        &= \begin{bmatrix}
            v_1\\v_2
        \end{bmatrix}
    \end{align}
    \item Let's generalize this concept to project one vector $\vec{w}$ on another vector $\vec{v}$
    \begin{center}
        \begin{tikzpicture}
            \draw[thick,->] (0,0) -- (8,2) node[right] {$\vec{v}$};
            \draw[thick,->] (0,0) -- (3,3) node[right] {$\vec{w}$};
            \draw[very thick,->] (0,0) -- (3.53,0.88) node[midway,below] {$\vec{u}$};
            \draw[dotted] (3,3) -- (3.53,0.88);
        \end{tikzpicture}
    \end{center}
    We can say that: $\vec{u}$ is the projection of $\vec{w}$ on $\vec{v}$. Based on the way we have constructed $\vec{u}$ we know it has certain properties:
    \begin{itemize}
        \item $\vec{u}$ is parallel to $\vec{v}$, so it can be expressed as a multiple of $\vec{v}$ such that:
        \begin{equation}
            \vec{u}=c\vec{v}
            \label{eq:}
        \end{equation}
        where $c$ is a scalar.
        \item We can say that $\vec{w}-\vec{u}$ (and by extension $\vec{u}-\vec{w}$) is orthogonal to $\vec{v}$:
        \begin{equation}
            (\vec{w}-\vec{u})\cdot \vec{v}=0
            \label{eq:}
        \end{equation}
    \end{itemize}
    Using these two properties, we can solve for the unknown $c$:
    \begin{align}
        (\vec{w}-\vec{u})\cdot \vec{v} &= 0 \\
        \vec{w}\cdot\vec{v}-\vec{u}\cdot\vec{v} &= 0 \\
        \vec{w}\cdot\vec{v}-(c\vec{v})\cdot\vec{v} &= 0 \\ 
        \vec{w}\cdot\vec{v}-c\left(\vec{v}\cdot\vec{v}\right) = 0 
    \end{align}
    Solving for $c$ gives:
    \begin{equation}
        c=\frac{\vec{w}\cdot\vec{v}}{\vec{v}\cdot\vec{v}}
        \label{eq:}
    \end{equation}
    so we have:
    \begin{equation}
        \boxed{\vec{u}=\frac{\vec{w}\cdot\vec{v}}{\vec{v}\cdot\vec{v}}\vec{v}}
        \label{eq:}
    \end{equation}
    \begin{definition}
        The projection of $\vec{w}$ on $\vec{v}$ can be written as:
        \begin{equation}
            \vec{u}=\text{proj}_{\vec{v}}\vec{w}
            =
            \frac{\vec{w}\cdot\vec{v}}{\vec{v}\cdot\vec{v}}\vec{v}
            = \frac{\vec{w}\cdot\vec{v}}{\lVert \vec{v} \rVert^2} \vec{v}
            = \frac{\vec{w}\cdot\vec{v}}{\lVert \vec{v} \rVert} \frac{1}{\lVert\vec{v}\rVert}\vec{v}
            \label{eq:}
        \end{equation}
        where the last part $\frac{1}{\lVert\vec{v}\rVert}\vec{v}$ is a unit vector pointing in the direction of $\vec{v}$.
    \end{definition}
    \item Suppose we wish to project a vector $\vec{v}$ onto another vector (e.g. z-axis) in three dimensions. \textit{The same formula applies.}
    \item Suppose instead we wish to project $\vec{v}$ on a plane, such as the xy plane? If $\vec{v}=\begin{bmatrix}
        v_1\\v_2\\v_3
    \end{bmatrix}$. Then the projection would be $\begin{bmatrix}
        v_1\\v_2\\0
    \end{bmatrix}$.
\end{itemize}