\section{Foundation}
\begin{itemize}
    \item Electric current describes the rate of \textit{positive} charge flow\footnote{\url{https://xkcd.com/567/}}:
    \begin{equation}
        i(t) = \frac{dq}{dt}
    \end{equation}
    and is measured in amperes $[A]$. The charge $Q$ that passes a given point is:
    \begin{equation}
        Q = \int_{t_0}^{t} i \dd{t}
    \end{equation}
    and is measured in coulombs $[C]$.
    \item Voltage describes the electric potential difference across the element:
    \begin{equation}
        V_{ab} = \frac{dW}{dq}
    \end{equation}
    which is physically described as the work required to move a unit positive charge of $+1\si{\coulomb}$ from $b$ to $a$. It is measured in volts $[V]$.
    \begin{center}
        \begin{tikzpicture}
            \draw (-2,0) node[left] {$a$} to [european resistor, v_={~}, l^=$V_{ab}$] (2,0) node[right] {$b$};
        \end{tikzpicture}
    \end{center}
    Notice that the below two setups are equivalent:
    \begin{center}
        \begin{tikzpicture}
            \draw (-2,0) node[left] {$a$} to [european resistor, v_={~}, l^=$10\si{\volt}$] (2,0) node[right] {$b$};
        \end{tikzpicture}\hspace{10mm}
        \begin{tikzpicture}
            \draw (2,0) node[right] {$b$} to [european resistor, v^={~}, l_=$-10\si{\volt}$] (-2,0) node[left] {$b$};
        \end{tikzpicture}
    \end{center}
    \item Power is the rate of energy flow:
    \begin{equation}
        P = V \cdot I
    \end{equation}
    Energy produced by the battery and the current generated runs through the bulb and the energy is dissipated into heat and light. Measured in watts $[W]$. We show this is true from the classical definition of power:
    \begin{equation}
        P = \frac{dW}{dt} = \frac{dW}{dq} \cdot \frac{dq}{dt} = V \cdot i
    \end{equation}
    \item In a DC (direct current) circuit, current flows in one direction and is constant with respect to time. In an AC (alternating current) circuit, the curret varies direction with respect to time.
    \item To distinguish between supplying or absorbing power, we use passive sign convention:
    \begin{definition}
        If power is positive, the element absorbs power and if negative, the element supplies power. Take the following circuit for example:
        \begin{center}
            \begin{tikzpicture}
                \draw (0,0) to [battery, v^={~}, l_=$9\si{\volt}$, invert] (0,3) to [short,i={$50\si{\milli\ampere}$}] (4,3) to [lamp] (4,0) to [short,i=$50\si{\milli\ampere}$] (0,0);
            \end{tikzpicture}
        \end{center}
        and we will get:
        \begin{align}
            P_\text{bulb} &= + iV = +0.45\si{\watt} ,\, \text{(absorbing)} \\ 
            P_\text{battery} &= - iV = -0.45\si{\watt} ,\, \text{(supplying)}
        \end{align}
        which gives:
        \begin{equation}
            \sum P = 0
        \end{equation}
    \end{definition}
\end{itemize}