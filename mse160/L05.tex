\section{X Ray Diffraction}
\begin{itemize}
    \item Suppose there is an x-ray source that reflects off a sample and hits a detector at a certain angle $\theta$. If we plot the intensity as a function of $\theta$, we see that it peaks at very specific values.
    \begin{center}
        \incfig{bragg}
    \end{center}
    \item This is a result of constructive interference where the peaks of both wavelengths add up at the detector.
    \item This will only occur if the extra distance (shown in yellow) is an integer multiple of the wavelength $\lambda$. Mathematically:
    \begin{equation}
        2d\sin\theta = n\lambda
    \end{equation}
    known as \textbf{Bragg's Law}, where $n$ is an integer.
    \item The interplanar spacing is related to the $hkl$ miller indices via:
    \begin{equation}
        d = \frac{a}{\sqrt{h^2+k^2+l^2}}
    \end{equation}
\end{itemize}