\section{Inelastic Behaviour}
\begin{itemize}
    \item Permanent deformation can be defined in three ways:
    \begin{enumerate}
        \item Upon unloading, sample does not return to original dimensions
        \item Strain does not return to zero
        \item Atoms move to new positions
        \item Occurs near the end of linear behaviour
    \end{enumerate}
    \item Plastic comes from the greek word \textit{plastikos}, which means to shape or to sculpt. In this course, plastic does not refer to the material type but instead the permanent deformation.
    \item The \textbf{strength} of a material describes when the permanent deformation occurs.
    \item The stress strain curve for different materials resemble different shapes.
    \begin{itemize}
        \item Polymers have a distinct yielding region
        \item Metals start a concave down behaviour as soon as elastic deformation ends
        \item Ceramics have linear behaviour all the way until they fracture
    \end{itemize}
    \item For polymers, the use of \textit{Young's Modulus} is misleading since the elastic behaviour depends on several different types of bonds, while Young's Modulus is related to the bahviour of a single bond. As a result, the term \textbf{elastic modulus} is used to describe polymers and composite materials.
\end{itemize}