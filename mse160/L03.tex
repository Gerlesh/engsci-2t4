\section{Structure Property Relationship}
\begin{itemize}
    \item If we plot the relationship between the Young's Modulus and the density, we find that there are no materials with a high Young's Modulus and a low density and that most materials fall on a single line.
    \item The \textbf{materials performance index} $MPI$ for a beam is defined as:
    \begin{equation}
        MPI = \frac{E^{1/2}}{\rho}
    \end{equation}
    \begin{proof}
        The deflection of a beam takes the form of:
        \begin{equation}
            \delta = \frac{FL^3}{CEA^2} \implies A = \left(\frac{FL^3}{CE\delta}\right)^{1/2}
        \end{equation}
        and using the equation:
        \begin{equation}
            m=AL\rho
        \end{equation}
        we are able to retrieve the equationa bove.
    \end{proof}
    \item If we were to plot the Young's Modulus against the density on a log-log scale and draw out the curve:
    \begin{equation}
    \log E = 2\log \rho + \log MPI    
    \end{equation}
    then materials on the same line would have the same materials performance index. If we shift this line to the left, the performance index of the material improves.
    \item For a beam, performing this analysis will give us composites such as carbon fibre. Ceramics also belong in this category but because they are susceptible to catasctophic brittle fracture, we can ignore them.
    \item For a loaded plate, the MPI is:
    \begin{equation}
        MPI = \frac{E^{1/3}}{\rho}
    \end{equation}
    and the best in this category is wood.
    \item \textbf{Polycrystalline} refers to materials made of crystals, typically around the micron scale.
    \item Most metals and some ceramics are crystalline (e.g. sapphire), but not everything is. \textbf{Amorphous} materials are not organized, an example being window glass.
    \begin{itemize}
        \item Long-range order refers to a organization at a distance well beyound the nearest neighbour
        \item Short-range order refers to organization at only the first or second nearest neighbour.
    \end{itemize}
    \item One such highly organized material has a \textbf{face-centered cubic} (FCC) structure. Many metals have this organization:
    \begin{figure}[ht]
        \centering
        \incfig{fcc}
    \end{figure}
    \item Each of the eight corners have $\frac{1}{8}$ of an atom and each of the six faces have $\frac{1}{2}$ of an atom, which gives: $n_\text{FCC}=4$ atoms in a unit cell.
    \item Given $n$, we can calculate the density of an FCC material such as alunium, via:
    \begin{equation}
        \rho = \frac{n A}{a^3N_A}
    \end{equation}
    where $a$ is the length of the unit cube known as the \textbf{lattice parameter}, $A$ is the molar mass, and $N_A$ is Avocado's number.
    \item Based off of the hard sphere model, we treat atoms in ordered solids as hard spheres that make contact with their nearest neighbours. However, this is very hard to draw so we will go by the \textit{reduced sphere} model by drawing atoms smaller.
    \begin{example}
        Suppose we wish to find the theoretical density of Aluminum, which is a FCC structure. We are given:
        \begin{align*}
            A &= \SI{26.892}{\gram\per\mole} \\ 
            R &= \SI{143}{\pico\meter}
        \end{align*}
        The lattice parameter is given by:
        \begin{equation}
            a = 2\sqrt{2}R
        \end{equation}
        which gives us:
        \begin{equation}
            \rho = \frac{nA}{(2\sqrt{2}R)^3 N_A} = \SI{2.71}{\gram\per\centi\meter\cubed}
        \end{equation}
    \end{example}
    \item The \textbf{atomic packing factor} is given by:
    \begin{equation}
        \text{APF} = \frac{V_\text{spheres}}{V_\text{unit cell}}
    \end{equation}
    Since the unit cell is a cube with volume $a^3$, we have:
    \begin{equation}
        \text{APF} = \frac{4\pi R^3 n}{3a^3}
    \end{equation}
    and for FCC:
    \begin{equation}
        \text{APF} = \frac{16\pi R^3}{3a^3}
    \end{equation}
    We can also relate the $R$ and $a$ together via:
    \begin{equation}
        a = 2\sqrt{2}R
    \end{equation}
    for FCC packing, which yields:
    \begin{equation}
        \text{APF}_\text{FCC} = \frac{\pi}{3\sqrt{2}} \approx 0.74
    \end{equation}
    which is the highest possible value. The maximum fraction that a volume can be filled up is $74\%$.
\end{itemize}