\section{Introduction}
\subsection{Types of Material}
\begin{itemize}
    \item There are three classes of material (though not all materials fall under these categories):
    \begin{itemize}
        \item Metals
        \item Ceramics 
        \item Polymers
    \end{itemize}
    \item \textbf{Metals} (e.g. Fe, Cr, Cu, Zn, Al) are held together with \emph{mellatic} bonds and is described by \textbf{bond theory}.
    \item \textbf{Ceramics} (e.g. poreclain, concrete) are held together with \emph{ionic} bonds and are \emph{brittle}. A lot of them are metal oxides.
    \item \textbf{Polymer} (Teflon\textregistered, Gore-tex\textregistered, polyethylene) \emph{tend} to be from \emph{covalent bonds}
    \begin{warning}
        The word plastic actually describes a material property, and not a material type. There are plastics that are not polymers.
    \end{warning}
    \item Examples of materials that do not fall under this classification scheme include wood, skin, superconductors, and more.
\end{itemize}
\subsection{Elastic Behaviour}
\begin{itemize}
    \item Hooke's law tells us that $F=-k\Delta x$, where $\Delta x$ is the displacement from equilibrium.
    \item \textbf{Engineering stress} is defined as $\sigma = \frac{F}{A_0}$ where $A_0$ is the \textit{initial} (unloaded) cross-sectional area.
    \begin{warning}
        Due to material properties, the cross sectional area of a spring can change as it elongates or compresses, so the engineering stress only refers to the initial cross sectyional area. The \textit{true stress} refers to the force divided by the real area.
    \end{warning}
    \item \textbf{Engineering strain} is defined as $\varepsilon = \frac{\Delta \ell}{\ell_0}$ and the two are related via the \textbf{Young's Modulus}:
    \begin{equation}
        \sigma = E\varepsilon
    \end{equation}
\end{itemize}