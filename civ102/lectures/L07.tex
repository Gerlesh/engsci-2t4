\section{Lecture 7: Power of Springing Bodies}
\begin{itemize}
    \item The force from an elastic spring is linear with respect to the disturbance:
    \begin{equation}
        |F|=k|\Delta x|
        \label{eq:}
    \end{equation}
    \item When a spring is perturbed from its equilibrium position, it will oscillate back and forth:
    \begin{center}
        \begin{tikzpicture}
        \begin{axis}[
        legend pos=outer north east,
        title=Oscillation of a Spring,
        axis lines = box,
        xlabel = $x$,
        ylabel = $y$,
        variable = t,
        trig format plots = rad,
        ]
        \addplot [
            domain=0:10,
            samples=70,
            color=blue,
            ]
            {cos(x)};
        \addlegendentry{$\cos(x)$}
        \draw[dotted] (0,0) -- (10,0);
        \draw[dotted] (6.28,0) -- (6.28,1) node[midway,left] {$A$};
        \end{axis}
        \end{tikzpicture}
    \end{center}
    and the period is given by:
    \begin{equation}
        T=2\pi\sqrt{\frac{m}{k}}
        \label{eq:}
    \end{equation}
    \item The differential equation is given by:
    \begin{equation}
        \frac{d^2y}{dt^2} = -\frac{k}{m}y-g
        \label{eq:}
    \end{equation}
    and comes from Newton's second law. The solution to this equation gives:
    \begin{equation}
        y=A\cos\left(\sqrt{\frac{k}{m}}t\right)
        \label{eq:}
    \end{equation}
    \item The natural frequency is given by:
    \begin{equation}
        f=\frac{1}{T}=\frac{1}{2\pi}\sqrt{\frac{k}{m}}
        \label{eq:}
    \end{equation}
    
\end{itemize}