\documentclass{article}
\usepackage{civ}

\title{CIV102: Problem Set \#2}
\author{QiLin Xue \\ \href{mailto:qilin.xue@mail.utoronto.ca}{qilin.xue@mail.utoronto.ca}}
\date{Fall 2020}
\usepackage{mathrsfs}
\usetikzlibrary{arrows}
\usepackage{siunitx}
\usepackage{wasysym}
\usetikzlibrary{calc}
\usepackage{xcolor}
\setlength\parindent{0pt}

\begin{document}
\maketitle
\section{Problem One}
\textbf{(a)} The weight of $N$ steel wires combined with the weight of a full lift is given by:
\begin{equation}
    \frac{Nw\pi d^2H}{4}+W_\text{lift}+20W_\text{person}
    \label{eq:1-max-weight}
\end{equation}
where $w=77\si{\kilo\newton\per\meter\cubed}$ is the weight density, $d=6\si{\milli\meter}$ is the diameter, and $H=5500\si{\meter}$ is the maximum length of the rope. The maximum force $N$ cables can support is given by:
\begin{equation}
    F_\text{max} = N\sigma_\text{max allowable} \frac{\pi d^2}{4}
    \label{eq:}
\end{equation}
where each cable is at its maximum allowable stress, and the weight is evenly distributed throughout. The tension of the cables at the top should support the maximum weight given in equation (\ref{eq:1-max-weight}), so we have:
\begin{align}
    N\sigma_\text{max allowable} \frac{\pi d^2}{4} &\ge \frac{Nw\pi d^2H}{4}+W_\text{lift}+20W_\text{person} \\ 
    N\frac{\pi d^2}{4}\left(\sigma_\text{max allowable}-wH\right) &\ge W_\text{lift}+20W_\text{person} \\ 
    N &\ge \frac{4(W_\text{lift}+20W_\text{person})}{\pi d^2 (\sigma_\text{max allowable}-wH)} \\ 
    N &\ge 5.2
    \label{eq:}
\end{align}
Since $N$ is an integer, we must have $N=6$.

\textbf{(b)} We do the same thing, except this time we want to solve for $H$ and replacing $\sigma_\text{max allowable}$ with $\sigma_\text{ultimate}$ to determine at what height will the cables break:
\begin{align}
    N\sigma_\text{ultimate} \frac{\pi d^2}{4} &= \frac{Nw\pi d^2H}{4}+W_\text{lift}+20W_\text{person} \\ 
    N\sigma_\text{ultimate} &= NwH+\frac{4}{\pi d^2}\left(W_\text{lift}+20W_\text{person}\right) \\ 
    H &= \frac{\sigma_\text{ultimate}}{w}-\frac{4\left(W_\text{lift}+20W_\text{person}\right)}{Nw\pi d^2} \\ 
    &= 14360\si{\meter}
\end{align}
where we have used $N=6$ from earlier. It makes sense that this is slightly more than double than the normal amount. Since the allowable stress on the wires is doubled, the cables can now support twice the weight as earlier. Since a lot of the weight is part of the actual lift and extending the cable further doesn't change that, the extra cable (past the double point) will make up for the weight of the lift.

\newpage
\section{Problem Two}
The ignore the mass of the cables. Assuming a typical weight of $w=77\si{\kilo\newton\per\meter\cubed}$, the weight of all three wires combined would be equal to:
\begin{equation}
    W = \frac{\pi d^2}{4} \left(1750\si{\milli\meter}+2000\si{\milli\meter}+2500\si{\milli\meter}\right) = 18.90 \times 10^{-3} \si{\kilo\newton}
    \label{eq:}
\end{equation}
which is on the order of $10^2$ smaller compared to the rest of the load, so we can safely ignore them.

Denote the strings from top to bottom from $1$ to $3$. Thus, we need to find $\Delta L_1+\Delta L_2+\Delta L_3$ and add it to the initial length of $AB$. We can calculate the length extension of the $n^\text{th}$ cable via:
\begin{align}
    E\frac{\pi d_n^2}{4} \frac{\Delta L_n}{L_n} &= W_n + W_\text{below} \\ 
    \Delta L &= \frac{4L_n(W_n + W_\text{below})}{E\pi d_n^2}
    \label{eq:}
\end{align}
We do this by taking everything underneath cable $n$ as a single system such that the cable supports a weight of $W_n+W_\text{below}$. Substituting in numbers, we get:
\begin{align}
    \Delta L_1 &= 5.5704 \si{\milli\meter}\\ 
    \Delta L_2 &= 2.8011 \si{\milli\meter}\\ 
    \Delta L_3 &= 7.0736 \si{\milli\meter}\\
\end{align}
So the total change in length is:
\begin{equation}
    \Delta L_\text{total} = 15.45\si{\milli\meter}
    \label{eq:}
\end{equation}
The total length of $AB$ is thus:
\begin{equation}
    L = 1750+300+2000+300+2500+300+15.45 = 7170 \si{\milli\meter}
    \label{eq:}
\end{equation}
The spring constant is given by $k\equiv \frac{YA}{L}$, so the energy stored in cable $n$ is given by:
\begin{align}
    U_n &=\frac{1}{2}\left(\frac{E\pi d_n^2}{4L_n}\right)\Delta L^2 \\ 
    &= \frac{1}{2}\left(\frac{E\pi d_n^2}{4L_n}\right)\left(\frac{4L_n(W_n + W_\text{below})}{E\pi d_n^2}\right)\Delta L \\ 
    &= \frac{1}{2}\left(W_n+W_\text{below}\right)\Delta L_n
    \label{eq:}
\end{align}
Substituting in numbers, we get:
\begin{align}
    U_1 &= 22.28 \si{\joule}\\ 
    U_2 &= 7.703 \si{\joule}\\ 
    U_3 &= 14.15 \si{\joule}\\
    \label{eq:}
\end{align}
so the total stored energy is:
\begin{equation}
    U_\text{total}=44.1\si{\joule}
    \label{eq:}
\end{equation}

\newpage
\section{Problem Three}
From the lecture, it was derived that the maximum tension shared by the cables is:
\begin{equation}
    T_\text{max} = \sqrt{\left(\frac{wL^2}{8h}\right)^2+\left(\frac{wL}{2}\right)^2}
    \label{eq:}
\end{equation}
where $w$ is the weight per unit length. If $D=36\si{\meter}$ is the width of the bridge, then $W\equiv w/D=27.5\si{\kilo\pascal}$ is the weight per unit area of the bridge. We can rewrite the equation as:
\begin{equation}
    T_\text{max} = \frac{WDL}{2}\sqrt{\frac{1}{16}\left(\frac{L}{h}\right)^2+1}
    \label{eq:}
\end{equation}
If each cable has $N$ wires, then the total number of wires supporting the bridge is $2N$. If the weight is shared equally, then they are all at the maximum allowable stress of $\sigma_\text{allowable}$. The total tension in the wires is thus:
\begin{equation}
    T = 2N\sigma_\text{allowable}\left(\frac{\pi d^2}{4}\right)
    \label{eq:}
\end{equation}
This tension force must be greater than the maximum needed, or:
\begin{align}
    2N\sigma_\text{allowable}\left(\frac{\pi d^2}{4}\right) &\ge \frac{WDL}{2}\sqrt{\frac{1}{16}\left(\frac{L}{h}\right)^2+1} \\ 
    N &\ge \frac{WLD}{\pi d^2\sigma_\text{allowable}}\sqrt{\frac{1}{16}\left(\frac{L}{h}\right)^2+1} \label{eq:max wires for 3}\\
    &\ge 76033
    \label{eq:}
\end{align}
Rounding up to three significant digits, we have $\boxed{N=76,100}$ wires on each cable. This calculation has assumed that the weight of the wires are negligible. Indeed, if we were to calculate the weight of the wires, it would give:
\begin{equation}
    W_\text{wires}= 2Nw_\text{wire}\pi d^2 L \approx 514 \times 10^6 \si{\newton}
    \label{eq:}
\end{equation}
making the crude approximation that the cable is horizontal, giving roughly 20\% that of the actual deck. We can estimate what first order correction we can add to our model. Suppose $w_\text{wire}=77\si{\kilo\newton\per\meter\cubed}$ is the weight density of the steel wires. The effective weight per unit area is thus:
\begin{equation}
    \frac{W_\text{wires}}{A} = \frac{2Nw_\text{wire}\pi d^2L}{DL}
    \label{eq:}
\end{equation}
and we can write the effective weight per unit area of the deck as:
\begin{equation}
    W' = W + \frac{2Nw_\text{wire}\pi d^2}{D} = 34.49\si{\kilo\pascal}
    \label{eq:adjusted pressure}
\end{equation}
and substituting this into equation (\ref{eq:max wires for 3}), we get:
\begin{equation}
    N \ge 95351
    \label{eq:}
\end{equation}
or $N=95400$. Note here that we have neglected the mass of $\Delta N = 19400$ wires, which is on the order of magnitude of $25\%$ less than the number of wires we ignored earlier and constitute $5\%$ of the mass of the deck, so it's safe to ignore this second correction for now.
\vspace{2mm}

We can come up with an even more accurate estimate by determining an expression for the exact length of the cable. Since the weight is uniformly distributed horizontally, the cable will take the shape of a parabola:
\begin{equation}
    y(x)=Ax^2
    \label{eq:}
\end{equation}
where $y(L/2)=h$. We can then solve for $A$ to be:
\begin{equation}
    A = \frac{4h}{L^2}
    \label{eq:}
\end{equation}
The arclength is given by:
\begin{equation}
    L'/2 =\int_0^{L/2}\sqrt{1+\left(\frac{dy}{dx}\right)^2}\dd{x}
    \label{eq:}
\end{equation}
where $\displaystyle \frac{dy}{dx}=\frac{8h}{L^2}x$, so substituting it in, we get:
\begin{equation}
    L'/2 = \int_0^{L/2}\sqrt{1+\left(\frac{64h^2}{L^4}\right)x^2}\dd{x}
    \label{eq:}
\end{equation}
This is a super ugly integral I do not know how to solve (probably some inverse trig substitution), but instead I will make the assumption that $h^2 \ll L^2$ and apply a first order binomial expansion to:
\begin{align}
    L'/2 &= \int_0^{L/2} 1+\frac{1}{2}\frac{64h^2}{L^4}x^2 \dd{x} \\ 
    &= \frac{L}{2}+\frac{1}{2}\frac{1}{3}\frac{64h^2}{L^4}\frac{L^3}{8} \\
    L' &= L + \frac{8}{3}\frac{h^2}{L} 
    \label{eq:}
\end{align}
It turns out this first order term (at least in $h^2$) is $\displaystyle \frac{8}{3}\frac{h^2}{L}\approx 54.22\si{\meter}$, which I did not figure out in the panicked half an hour on my first civ quiz. Sorry for that. The actual length of the table is then $L'=2045\si{\meter}.$ Combining this new length with equation (\ref{eq:adjusted pressure}) amd (\ref{eq:max wires for 3}), we get:
\begin{equation}
    N \ge \left(W + \frac{2Nw_\text{wire}\pi d^2}{D}\right)\frac{L'D}{\pi d^2\sigma_\text{allowable}}\sqrt{\frac{1}{16}\left(\frac{L}{h}\right)^2+1}
    \label{eq:}
\end{equation}
Solving this for $N$ gives me $N \ge 109042$ cables, which rounds up to $N=110,000$ cables.

\newpage
\section{Problem Four}
\subsection{Part A}
\textbf{(i)} The force applied on the concrete does not change, so the temperature change doesn't affect the stress of:
\begin{equation}
    \sigma_P = \frac{P}{A} = 2.29 \si{\mega\pascal}
    \label{eq:}
\end{equation}
and ignoring effects from temperature, the strain caused by this force is given by:
\begin{equation}
    \epsilon_\sigma = \frac{\sigma_P}{E}=\frac{2.29}{30000} = 7.62 \times 10^{-5}
    \label{eq:}
\end{equation}
where we have used $E=30,000\si{\mega\pascal}$ for the Young's Modulus of concrete.

\textbf{(ii)} The thermal strain due to the change in temperature of $\Delta T = +15\si{\celsius}$ is:
\begin{equation}
    \epsilon_\text{th}=\alpha \Delta T = 9 \times 10^{-6} \cdot 15 = 13.5 \times 10^{-5}
\end{equation}
where $9 \times 10^{-6} \si{\per\celsius}$ is used for the coefficient of thermal expansion. Under a free expansion, no stresses will be experienced.

\textbf{(iii)} The total strain is given by:
\begin{equation}
    \epsilon_\text{total}+\epsilon_\sigma+\epsilon_\text{th} = 21.1 \times 10^{-5}
    \label{eq:}
\end{equation}
since the total strain is also equal to $\Delta l/l_0$, we can isolate and solve for $\Delta l$ to be:
\begin{equation}
    \Delta l = l_0\epsilon_\text{total} = 211\si{\micro\meter}
    \label{eq:}
\end{equation}
\subsection{Part B}
We can use symmetry to explain this. Cooling the block and then placing it in between the walls is the same as placing it in between the walls and then cool it. We will look at the first case. When the concrete is cooled, it will contract. The molecules will on average move slower and take up less space. This can be seen in the positive coefficient of thermal expansion:
\begin{equation}
    \Delta L= L\alpha\Delta T
    \label{eq:}
\end{equation}
If $\Delta T$ is negative, then the change in length will also be negative. The block of concrete will now be shorter than $1000\si{\milli\meter}$, so in order for the ends to be fixed to the walls, the block will need to be stretched causing it to develop tensile stresses. The main idea is that \textbf{the strain caused by the drop in temperature is opposite to the strain caused by the tensile stresses}. We are able to make this assumption (which isn't entirely accurate) since the change in length is so much smaller than $1000\si{\milli\metre}$. The thermal contraction actually changes the equilibrium length but for all purposes, that can be ignored. Thus, the ultimate stress is given by:
\begin{equation}
    \sigma_\text{ultimate} = E|\epsilon_\text{th}| = E\alpha |\Delta T|
    \label{eq:}
\end{equation}
solving for $\Delta T$, we get:
\begin{equation}
    |\Delta T| = \frac{\sigma_\text{ultimate}}{E\alpha} = \frac{3}{30,000\cdot (9\times 10^{-6})} = 11.11\si{\celsius}
    \label{eq:}
\end{equation}
Thus, the temperature needs to decrease by only $11.11\si{\celsius}$ for the concrete to break.

Note that this problem assumed the expansion/contraction was only in one dimension. This assumption is valid because the change in the cross sectional area is proportional to second order in $\alpha$ (e.g. $\alpha^2$), and since $\alpha \ll 1$, we can effectively ignore it.

\newpage
\section{Problem Five}
\textbf{(a)} We can draw a diagram:
\begin{center}
    \begin{tikzpicture}
        \coordinate (A) at (-5,2);
        \coordinate (C) at (5,2);
        \coordinate (B) at (0,0);
        
        \draw[] (B) -- (A);
        \draw[] (B) -- (C);

        \draw[fill=black] (B) circle (0.05) node[above] {$B$};
        \draw[fill=black] (A) circle (0.05) node[above] {$A$};
        \draw[fill=black] (C) circle (0.05) node[above] {$C$};

        \draw[<->,dotted] (-5,2.5) -- (5,2.5) node[midway,above] {$60\si{\meter}$};
        \draw[<->,dotted] (-5.5,2) -- (-5.5,0) node[midway,right] {$6\si{\meter}$};

        \draw[color=red,thick,->] (B) -- (0,-2) node[right] {$W=2000\si{\newton}$};
        \draw[color=blue,thick,->] (B) -- ($(B)!0.5!(C)$) node[below] {$T$};
        \draw[color=blue,thick,->] (B) -- ($(B)!0.5!(A)$) node[below] {$T$};
    \end{tikzpicture}
\end{center}
We already spent a lot of time working with these simple free body diagrams, so I'll move fast through this. By symmetry, the tension in the two upper strings are the same, so we can call them both $T$. The angle $T$ makes with the horizontal is given by:
\begin{equation}
    \theta = \tan^{-1} \left(\frac{6}{60/2}\right) = \tan^{-1}\left(\frac{1}{5}\right) = 11.31^\circ
    \label{eq:}
\end{equation}
Then balancing forces in the vertical direction gives:
\begin{equation}
    F_\text{net,vertical} = 0 \implies 2T\sin\theta = W \implies T = \frac{W}{2\sin\theta} \implies \boxed{T =5.10 \si{\kilo\newton}}
    \label{eq:tension}
\end{equation}
The tensile stress in each rope is
\begin{equation}
    \sigma = \frac{4T}{\pi d^2} \implies \boxed{\sigma = 721 \si{\mega\pascal}}
\end{equation}
and the safety factor is thus:
\begin{equation}
    f=\frac{1500}{\sigma}\implies \boxed{f = 2.08}.
    \label{eq:}
\end{equation}
Due to this tension, the change in length of each rope is given by:
\begin{equation}
    \Delta L = \frac{L_0T}{AE}
    \label{eq:change in length}
\end{equation}
Using $L_0=30\si{\meter}$, $A=\frac{\pi d^2}{4}$ with $d=3\si{\milli\meter}$, and $E=200,000\si{\mega\pascal}$, then:
\begin{equation}
    \boxed{\Delta L = 0.1103\si{\meter}}
    \label{eq:}
\end{equation}
To find the change in height of the weight, we can draw a diagram:
\begin{center}
    \begin{tikzpicture}
        \coordinate (A) at (-5,2);
        \coordinate (C) at (5,2);
        \coordinate (B) at (0,0);
        \coordinate (B') at (0,-2);
        \coordinate (P) at ($(A)!(B)!(B')$);

        \draw[] (B) -- (A);
        \draw[] (B) -- (B');
        \draw[] (B') -- (A);

        \draw[fill=black] (B') circle (0.05) node[right] {$B'$};
        \draw[fill=black] (B) circle (0.05) node[above] {$B$};
        \draw[fill=black] (A) circle (0.05) node[above] {$A$};
        \draw[fill=black] (P) circle (0.05) node[left] {$P$};

        \draw[<->,dotted] (-5,2.5) -- (0,2.5) node[midway,above] {$30\si{\meter}$};
        \draw[<->,dotted] (-5.5,2) -- (-5.5,0) node[midway,right] {$6\si{\meter}$};

        \draw[dotted] (P) -- (B);
    \end{tikzpicture}
\end{center}
Assuming that $BB' \ll AB$, then we can say that $\overline{AB}$ and $\overline{AB'}$ make the same angle with the horizontal, which results in $\angle PBB'=\theta$ and $\displaystyle BB'=\frac{B'P}{\sin\theta}$. Again, we can assume that $AB\approx AP$ which gives us so the change in length of the rope is equal to $\Delta L=B'P$. Therefore:
\begin{equation}
    \Delta h = BB'=\frac{\Delta L}{\sin\theta}=0.563\si{\meter}
    \label{eq:}
\end{equation}
We can get an exact answer (but it's probably not needed since the assumptions we are making in the problem statement are already very wild) by using sine law. We know that $\angle ABB' = 90^\circ+\theta$ so:
\begin{equation}
    \frac{\sin(90^\circ+\theta)}{AB'}=\frac{\sin(\angle AB'B)}{AB}
    \implies \angle AB'B = \sin^{-1}\left(\frac{L \cos\theta}{L+\Delta L}\right) = 77.7^\circ
    \label{eq:}
\end{equation}
Since the angles in a triangle sum up to $180^\circ$, we have: $\angle BAB' = 0.987^\circ$, and applying the cosine law:
\begin{equation}
    BB' = \Delta h = \sqrt{L^2+(L+\Delta L)^2-2L(L+\Delta L)\cos(0.987^\circ)} \implies \boxed{\Delta h= 0.539\si{\meter}}
    \label{eq:long ass}
\end{equation}
\textbf{(b)} We can just apply equation (\ref{eq:tension}) with the new angle
\begin{equation}
    \theta'=90^\circ-\angle AB'B=12.297^\circ
    \label{eq:}
\end{equation}
which gives:
\begin{equation}
    T' = \frac{W}{2\sin\theta'}\implies \boxed{T' = 4.70\si{\kilo\newton}}
    \label{eq:}
\end{equation}
and
\begin{equation}
    \sigma' = \frac{4T'}{\pi d^2} \implies \boxed{\sigma' = 664\si{\mega\newton}}
    \label{eq:}
\end{equation}
Therefore, the factor of safety becomes:
\begin{equation}
    f' = \frac{1500}{\sigma'} \implies \boxed{f = 2.26}.
    \label{eq:}
\end{equation}
The factor of safety went up, since the tension decreases! Intuitively, since the weight sags down, the wires are more ``vertical'' and as a result is easier for the wires to supply a vertical force to maintain the mass in equilibrium.

\textbf{(c)} Again, we can apply equation (\ref{eq:change in length}) to get the change in length as:
\begin{equation}
    \Delta L' = \frac{L_0 T'}{AE} = 0.1016\si{\meter}
\end{equation}
so the total length of each rope is
\begin{equation}
    L'' = L + \Delta L' \implies \boxed{L''= 30.7\si{\meter}}
    \label{eq:}
\end{equation}
It makes sense that $L''<L'$ since we essentially overestimated the strain beforehand. Since the stress decreases, it makes sense for the length of the rope to decrease as well.

We can plug this value of $\Delta L'$ into equation (\ref{eq:long ass}) to get:
\begin{equation}
    BB' = \Delta h = \sqrt{L^2+(L+\Delta L')^2-2L(L+\Delta L')\cos\left(\sin^{-1}\left(\frac{L \cos\theta}{L+\Delta L'}\right)\right)} \implies \boxed{\Delta h = 0.498\si{\meter}}
    \label{eq:}
\end{equation}

\textbf{(d)} When the strain need to be carefully controlled, it becomes dangerous to ignore second-order effects. This is especially true if the safety factor is already extremely high and the material isn't at risk of breaking. However, the safety factor is a misleading term as it says nothing about the true ``safety'' of a system. Just because a certain member won't snap doesn't make a design safe.

For example, a company that produces bungee jumping ropes would need to be especially careful. If they underestimate the strain, the person could hit the ground very fast. If they overestimate the strain, the person will stop faster, meaning they may experience a higher g-force than they expected.
\end{document}
