\section{Volumes}
\begin{itemize}
    \item Suppose we have a cylinder whose axis is parallel to the $x$ axis. Then we can break up the volume into thin sections:
    \begin{equation}
        V_i \simeq A_i \Delta x_i 
        \label{eq:}
    \end{equation}
    so the volume is:
    \begin{equation}
        V = \int_a^b A(x) \dd{x}
        \label{eq:}
    \end{equation}
    which is the general formula for the volume of any shape. If we can figure out $A(x)$ and the necessary bounds, we can find the volume for any change.
    \begin{example}
        Suppose we wish to find the volume of a sphere. If we slice it up into small disks, then the area of each disk is:
        \begin{equation}
            A=\pi\left(\sqrt{r^2-x^2}\right)^2
            \label{eq:}
        \end{equation}
        then the volume is:
        \begin{equation}
            V=\int_{-r}^r \pi (r^2-x^2)\dd{x} = \frac{4}{3}\pi r^3
            \label{eq:}
        \end{equation}
    \end{example}
    \item For \textbf{solids of revolution}, imagine rotating the curve around the x-axis. Then the area of each disk is:
    \begin{equation}
        A(x) = \pi f(x)^2
        \label{eq:}
    \end{equation}
    so using the disk method gives us the volume as:
    \begin{equation}
        V = \int_a^b \pi f(x)^2 \dd{x}
        \label{eq:}
    \end{equation}
    when the function $f(x)$ is rotated around the $x$ axis.
    \item This even works for curves that cross the axis, since we end up squaring $f(x)$. For example, we can apply the same formula to $y=2-x^2$ from $x\in[0,2]$.
    \item Additionally, we can use the disk method about the y-axis:
    \begin{equation}
        V=\int_c^d \pi g(y)^2 \dd{y}
        \label{eq:}
    \end{equation}
    \item If we want the volume of revolution formed by rotating the region between two curves. The area between the two curves $f(x)$ and $g(x)$ is:
    \begin{equation}
        A = \pi \left(f(x)^2-g(x)^2\right)
        \label{eq:}
    \end{equation}
    and we get:
    \begin{equation}
        V = \int_a^b \pi \left(f(x)^2-g(x)^2\right)\dd{x}
        \label{eq:}
    \end{equation}
    known as the washer method. It works similarly for regions rotated about the $y$ axis. Another way of looking at it is determining the volume of $f(x)$ rotated around the $x$ axis and subtracting the volume of $g(x)$ rotated around the $x$ axis from it.
    \item If we wish to find the volume bounded by two curves, then we need to find their intersection points first.
\end{itemize}