\section{Lecture 25: The Natural Logarithm}
\begin{itemize}
    \item The logarithm is defined as:
    \begin{definition}
        A logarithm function is a nonconstant differentiable function $f$ defined for $x\in \{\mathbb{R},(0,\infty)\}$ such that for all $a>0$ and $b>0$:
        \begin{equation}
            f(a\cdot b) = f(a)+f(b)
            \label{eq:}
        \end{equation}
    \end{definition}
    \item It has the following properties:
    \begin{itemize}
        \item $f(1)=0$
        \item $f(1/x)=-f(x)$
        \item $f(x/y)=f(x)-f(y)$
        \item $f'(x)=\frac{1}{x}f'(1)$.
    \end{itemize}
    The first three are trivial to prove, but the last one is not obvious:
    \begin{prooof}
        Define:
        \begin{align}
            f'(x)&=\lim_{h\to 0} \frac{f(x+h)-f(x)}{h} \\ 
            &= \lim_{h\to 0} \frac{1}{h}f\left(\frac{x+h}{x}\right) \\ 
            &= \lim_{h\to 0} \frac{f(1+\frac{h}{x})-f(1)}{x\cdot \frac{h}{x}} 
            \label{eq:}
        \end{align}
        Let $k\equiv \frac{h}{x}$ such that the limit becomes:
        \begin{equation}
            = \frac{1}{x} \lim_{k\to 0} \frac{f(1+k)-f(1)}{k}
            \label{eq:}
        \end{equation}
        where the right hand side gives the derivative of $f(x)$ evaluated to $f'(1)$. If we choose a function $f'(1)=1$, then it is known as the natural logarithm.
    \end{prooof}
    \item This leads to the definition of the natural logarithm:
    \begin{equation}
        \ln(x) = \int_1^x \frac{\dd{t}}{t}
        \label{eq:}
    \end{equation}
    for $x>0$. It has the following properties:
    \begin{itemize}
        \item $\ln(x)$ is defined on $(0,\infty)$. For $x>0$, it is strictly increasing.
        \item $\ln(x)$ is continuous, since it is differentiable.
        \item For $x>1$, $\ln(x)>0$.
        \item For $0<x<1$, $\ln(x)<0$.
        \item $\ln(a\cdot b) = \ln(a)+\ln(b)$.
        \item $\ln(x^{p/q}) = \frac{p}{q}\ln(x)$.
        \item The range of $\ln(x)$ is from $(-\infty,\infty)$.
        \item There is a number $e$ such that $\ln(e)=1$.
        \item $\ln(e^{p/q})=p/q$.
        \item For convention $\ln(x)=\log_e(x)$.
        \item $(\ln x)' = \frac{1}{x}$, increasing.
        \item $(\ln x)'' = -\frac{1}{x^2}$, concave down.
    \end{itemize}
\end{itemize}