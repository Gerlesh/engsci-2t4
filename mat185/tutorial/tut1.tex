\documentclass{article}

\usepackage{qilin}
\usepackage{multicol}

\title{MAT185 Tutorial 1}
\author{QiLin Xue}
\usepackage{bm}
\date{\today}

\begin{document}

\maketitle
\textit{Note:} The treatment of these tutorial questions are not always very rigorous. The general ideas however for a completely rigorous proof are provided and should not be difficult to complete.
\section{Tutorial Problems}
\subsection*{Problem One}
\textbf{(a)} The vector space must be $\{\bm{x}\}$ where $\bm{x}=\bm{0}$. This is because the zero vector belongs in all vector spaces, and if this space only has one vector, then it must be the zero vector.
\vspace{2mm}

\noindent \textbf{(b)} We have:
\begin{align}
    c\bm{x} &= (c+d-d)\bm{x} \\ 
    &= (c-d+d)\bm{x} \\ 
    &= (c-d)\bm{x} + c\bm{x} & (DSA) \\
\end{align}
We have a vector $\bm{v} \equiv (c-d) \bm{x}$ such that:
\begin{equation}
    \bm{v} + c\bm{x} = c\bm{x} 
\end{equation}
Per proposition 4, we must have $\bm{v} = \bm{0}$. Since $\bm{x}$ can be nonzero, then this means that $(c-d)=0$ (Z) or $c=d$. Alternatively, we could arrive at this in a much easier way using the cancellation theorem.
\vspace{2mm}

\noindent \textbf{(c)} Proof by contradiction: Suppose for the sake of contradiction that a vector space $V$ consists of $N$ distinct vectors with $N>0$. Per SC, if $\bm{x} \in V$, then $\lambda \bm{x} \in V$ with $\lambda \in \mathbb{R}$. However, there are infinite possible values of $\lambda$. We now need to show that $\lambda_1 \bm{x} \neq \lambda_2 \bm{x}$ if $\bm{x}\neq \bm{0}$ and $\lambda_1 \neq \lambda_2$. From (b), we have determined that if the two vectors are equal, then it must demand that $\lambda_1 = \lambda_2$, which isn't satisfied here, and thus we have found a contradiction. The vector space can however consist of one vector, the zero vector.
\subsection*{Problem Two}
\textbf{(i)} Since we are using normal addition and scalar multiplication, then the addition and multiplication axioms are satisfied. We now need to show that this is closed. Let $x=\frac{a}{b}$ and $y=\frac{c}{d}$ with $b,d \neq 0$ and $\gcd(a,b), \gcd(b,d) \neq 0$. We need to show that their sum is also a rational number:
\begin{equation}
    x+y = \frac{ad+cb}{bd}
\end{equation}
and since it can be written as a fraction, $x+y$ is also rational. Note that from our earlier condition, $bd\neq 0$. Similarly for scalar multiplication by $\lambda$, we have:
\begin{equation}
    \lambda x = \frac{\lambda a}{b}
\end{equation}
which is also a fraction.
\vspace{2mm}

\noindent \textbf{(ii)} We first make sure that the space is closed under addition:
\begin{equation}
    \bm{A} + \bm{B} =
    \begin{bmatrix}
    1 & 0 \\ 
    a & 1
    \end{bmatrix}
        \begin{bmatrix}
    1 & 0 \\ 
    b & 1
    \end{bmatrix}
    = 
    \begin{bmatrix}
    1 & 0 \\ 
    a+b & 1
    \end{bmatrix}
\end{equation}
Addition only changes the bottom left entry, and modifies it under normal addition rules. As a result, both addition closure and addition axioms will be satisfied since the set of all real numbers is a vector space. Similarly, scalar multiplication only affects the bottom left corner in the regular way, so this is a vector space.
\vspace{2mm}

\noindent  \textbf{(iii)} So many things are violated here! Take associativity of addition for example:
\begin{align}
    (\bm{x} + \bm{x}) + \bm{y} &= \bm{y} + \bm{y} \\ 
    &= \bm{x}
\end{align}
but:
\begin{align}
    \bm{x} + (\bm{x}+\bm{y}) &= \bm{x} + \bm{y} \\ 
    &= \bm{y}
\end{align}
\subsection*{Problem Three}
We first note that $a_1=b_1 \equiv \lambda_1$ and $a_2=b_2 \equiv \lambda_2$, otherwise commutativity  does not hold. Also notice that the first index is independent from the second index during addition and scalar multiplication. As a result, we consider the simpler problem. Is the following a vector space?
\begin{align}
    \bm{x} + \bm{y} = a(x+y)
\end{align}
From associativity, we have:
\begin{align}
    (c+d)\bm{x} = c\bm{x}+d\bm{x} = \lambda_1(cx+dx) = \lambda_1(c+d)x
\end{align}
but also
\begin{align}
    (c+d)\bm{x} = (c+d)x
\end{align}
which implies that $\lambda_1=\lambda_2 = a_1=b_1=a_2=b_2=1$.
\subsection*{Problem Four}
We only need to worry about closure since regular matrices are a vector space. We can tell that the vector space is closed under addition. Let $w(A)$ be the weight of a magic square $A \in \mathbb{M}_n$, then $w(A+B)=w(A)+w(B)$. Formally, let $\bm{A} = [a_{ij}]$ and $\bm{B} = [b_{ij}]$. Then:
\begin{align}
    \bm{A} + \bm{B} = [a_{ij}+b_{ij}]
\end{align}
The sum of the $m^\text{th}$ row of a magic square $\bm{A}$ is given by $S_m(\bm{A})$:
\begin{equation}
    S_m(\bm{A}) = \sum_{k=1}^n a_{mk}
\end{equation}
and thus:
\begin{equation}
    S_m(\bm{A}+\bm{B}) = \sum_{k=1}^n (a_{mk}+b_{mk}) = \sum_{k=1}^n a_{mk} + \sum_{k=1}^n a_{nk} = S_m(\bm{A}) + S_m(\bm{B})
\end{equation}
and similar reasoning for the columns and diagonals. We can also show that this is closed under scalar multiplication:
\begin{equation}
    S_m(\lambda \bm{A} ) =\sum_{k=1}^n \lambda a_{mk} = \lambda \sum_{k=1}^n a_{mk} = \lambda S_m(\bm{A})
\end{equation}
\vspace{2mm}

\noindent \textbf{Remarks:} It's interesting that this is almost closed under matrix multiplication, we can represent multiplication of two matrices by:
\begin{equation}
    \bm{A}\bm{B} = \left[\sum_{p=1}^n a_{ip}b_{pj}\right] 
\end{equation}
and thus:
\begin{equation}
    S_m\left(\bm{A}\bm{B}\right) = \sum_{k=1}^n \sum_{p=1}^n a_{mp}b_{pk}
\end{equation}
We can interchange the sums:
\begin{align}
    &= \sum_{p=1}^n \sum_{k=1}^n a_{mp}b_{pk} \\ 
    &= \sum_{p=1}^n \left(a_{mp} \sum_{k=1}^n b_{pk}\right) \\ 
    &= \sum_{p=1}^n \left(a_{mp} \cdot S_p(\bm{B}) \right) \\ 
    &= S_p(\bm{B}) \cdot \sum_{p=1}^n a_{mp} \\ 
    &= S_p(\bm{B}) \cdot S_m(\bm{B})
\end{align}
However for a magic square, $S_i=S_j$ for any $1 \le i,j \le n$ so this means that:
\begin{align}
    S_m(\bm{A}\bm{B}) = S_m(\bm{A}) S_m(\bm{B})
\end{align}
In words, this means that for \textit{any} row in the product $\bm{A}\bm{B}$, the sum will equal to the product of the weights of the original two matrices. However, this result is not true for columns or diagonals.
\end{document}
