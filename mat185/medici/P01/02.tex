% \begin{sol}
% We solve for the most general case via induction, that if $\bm{A}_1\bm{A_2} \cdots \bm{A_n}$ is invertible, then each of $\bm{A}_1,\bm{A}_2,\dots,\bm{A}_n$ are invertible. We start with the base case of $n=2$:
% Let $\bm{P}\equiv \bm{A}_1\bm{A_2}$ such that:
% \begin{align}
%     \bm{A}_1\bm{A_2}\bm{P}^{-1} &= \bm{I} & \text{(given)} \\ 
%     \bm{A}_1\left(\bm{A_2}\bm{P}^{-1}\right) &= \bm{I} & \text{(associativity)}
% \end{align}
% This shows that $\bm{A}_1$ is invertible and its inverse is equal to $\bm{A}_1^{-1} = \bm{A_2}\bm{P}^{-1}$. We can then multiply both sides by $\bm{A}^{-1}$ from the left side:
% \begin{align}
%     \bm{A_2}\bm{P}^{-1} &= \bm{A_1}^{-1} \\ 
%     \bm{A_2}\bm{P}^{-1}\bm{A}_1 &= I \\ 
%     \bm{A_2}\left(\bm{P}^{-1}\bm{A}_1\right) &= I
% \end{align}
% and by the same reasoning, $\bm{A_2}$ is also invertible. We then suppose that this is true for $n=k$ and we want to show it also holds true for $n=k+1$. We let:
% \begin{align}
%     \bm{A}_1\bm{A_2}\cdots\bm{A}_k\bm{A}_{k+1} &= \bm{P} \\ 
%     \bm{A}_1\bm{A_2}\cdots\bm{A}_k\bm{A}_{k+1}\bm{P}^{-1} &= \bm{I} \\
%     \bm{A}_{k+1}\bm{P}^{-1} &= \left(\bm{A}_1\bm{A_2}\cdots\bm{A}_k\right)^{-1} \\
%     \bm{A}_{k+1}\bm{P}^{-1}\left(\bm{A}_1\bm{A_2}\cdots\bm{A}_k\right) &= \bm{I} \\ 
%     \bm{A}_{k+1}\left(\bm{P}^{-1}\left(\bm{A}_1\bm{A_2}\cdots\bm{A}_k\right)\right) &= \bm{I}
% \end{align}
% We have shown that $\bm{A}_{k+1}$ is invertible, and from our assumption earlier that the original statement holds true for $n=k$, then the statement must also hold true for $n=k+1$. However, we proved that this was true for $n=2$, so by induction, it must hold for all $n$, including $n=3$.
% \end{sol}

%%% NOTE: I don't actually think the above proof is valid.

\begin{sol}
We define the invertible matrix $\bm{P}=\bm{A}\bm{B}\bm{C}$ such that:
\begin{align}
    \bm{A}\bm{B}\bm{C}\bm{P}^{-1} &= \bm{I} \\ 
    \bm{A}\underbrace{\left(\bm{B}\bm{C}\bm{P}^{-1}\right)}_{\bm{A}^{-1}} &= \bm{I} \\
\end{align}
\begin{align}
    \left(\bm{B}\bm{C}\bm{P}^{-1}\right) &= \bm{A}^{-1} \\ 
    \left(\bm{B}\bm{C}\bm{P}^{-1}\right)\bm{A} &= \bm{I} \\ 
    \bm{B}\underbrace{\left(\bm{C}\bm{P}^{-1}\bm{A}\right)}_{\bm{B}^{-1}} &= \bm{I} 
\end{align}
\begin{align}
    \left(\bm{C}\bm{P}^{-1}\bm{A}\right) &= \bm{B}^{-1} \\ 
    \bm{C}\underbrace{\left(\bm{P}^{-1}\bm{A}\bm{B}\right)}_{\bm{C}^{-1}} &= \bm{I}
\end{align}
and we are done.
\end{sol}