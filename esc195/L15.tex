\section{Series}
\begin{itemize}
    \item Suppose we wish to add the infinite series:
    \begin{equation}
        I = \frac{1}{2} + \frac{1}{4} + \frac{1}{8} + \frac{1}{16} + \cdots
    \end{equation}
    \item We can define the partial sum to be:
    \begin{align}
        s_0 &= a_0 = \sum_{k=0}^0 a_n \\ 
        s_1 &= a_0 + a_1 = \sum_{k=0}^1 a_n \\ 
        s_2 &= a_0 + a_1 + a_2 = \sum_{k=0}^2 a_n \\ 
        &\vdots \\ 
        s_n &= a_0+a_1+\cdots+a_n = \sum_{k=0}^n a_k
    \end{align}
    \item We can then consider the sequence $\{s_n\} = \{a_0, a_0+a_2, a_0+a_1+a_2, \cdots\}$. This sequence converges if the sum converges. Specifically, if $\lim_{n\to\infty}\{s_n\}=L$, then $\sum_{k=0}^\infty a_n = L$.
    \begin{example}
        Suppose we wish to evaluate:
        \begin{equation}
            \sum_{k=0}^\infty \frac{1}{(k+2)(k+3)} = \frac{1}{6}+\frac{1}{12}+\frac{1}{20}
        \end{equation}
        We can use partial fractions to write:
        \begin{equation}
            \frac{1}{(k+2)(k+3)} = \frac{1}{k+2} - \frac{1}{k+3}
        \end{equation}
        so the sum becomes:
        \begin{equation}
            =\frac{1}{2}-\frac{1}{3}+\frac{1}{3}-\frac{1}{4}+\frac{1}{4}+\cdots -\frac{1}{n+3} = \frac{1}{2}-\frac{1}{n+3}
        \end{equation}
        which is known as a telescoping sequence. Taking the limit as $n\to\infty$, we get that the sum converges to $\frac{1}{2}$.
    \end{example}
    \item The sum of a geometric series is:
    \begin{equation}
        x^0+x^1+x^2+x^3+\cdots = \sum_{k=0}^\infty x^k = \frac{1}{1-x}
    \end{equation}
    which converges when $|x|<1$.
    \begin{proof}
        Let $S_n = 1 + x + x^2 + \cdots + x^n$ and $xS_n = x + x^2 + x^3 + \cdots + x^{n+1}$. Then subtracting the two, we get:
        \begin{equation}
            S_n-xS_n = 1-x^{n+1} \implies S_n = \frac{1-x^{n+1}}{1-x}
        \end{equation}
        and for $|x|<1$, the limit gives us $\frac{1}{1-x}$ and if $|x|>1$, the limit diverges.
    \end{proof}
    \item Suppose we wish to write the repeating fraction as a decimal: $0.\overline{285714}$. This is equal to:
    \begin{align}
        &= \frac{28574}{10^6} + \frac{285714}{10^12} + \cdots \\ 
        &= \frac{28574}{10^6}\left(1+\frac{1}{10^6}+\frac{1}{10^{12}}+\cdots \right)
    \end{align}
    Evaluating this infinite series, we get:
    \begin{equation}
        \frac{2}{7}
    \end{equation}
    \begin{example}
        Suppose we wish to write out $\frac{x}{4-x^2}$ as a sum for $|x|<2$. We have:
        \begin{align}
            \frac{x}{4-x^2} &= \frac{x}{4}\left(\frac{1}{1-x^2/4}\right) \\ 
            &= \frac{x}{4} \sum_{k=0}^\infty \left(\frac{x^2}{4}\right)^k \\ 
            &= \frac{x}{4} \sum_{k=0}^\infty \left(\frac{x}{2}\right)^{2k} \\ 
            &= \frac{1}{2}\left[\frac{x}{2}+\left(\frac{x}{2}\right)^3+\left(\frac{x}{2}\right)^5+\cdots\right]
        \end{align}
    \end{example}
    \begin{theorem}
        Here are a few important properties that arise when applying limit laws:
        \begin{itemize}
            \item If $\sum_{k=0}^\infty a_k = n$ and $\sum_{k=0}^\infty b_k = M$, then $\sum_{k=0}^\infty (a_k+b_k) = L+M$.
            \item If $\sum_{k=0}^\infty a_k = L$, then $\sum_{k=0}^\infty \alpha a_k = \alpha L$ for $\alpha \in \mathbb{R}$.
        \end{itemize}
    \end{theorem}
    \begin{theorem}
        If $\sum_{k=0}^{\infty} a_k$ converges iff $\sum_{k=j}^\infty a_k$ converges where $j$ is a positive integer.
    \end{theorem}
    \begin{example}
        Suppose we are given that $\sum_{k=4}^\infty \frac{3^{k-1}}{3^{3k+1}}$ converges, then $\sum_{k=0}^\infty \frac{3^{k-1}}{3^{3k+1}}$ converges.
    \end{example}
    \begin{theorem}
        If $\sum_{k=0}^\infty a_k$ converges, then $a_k \to 0$ as $k\to\infty0$. 
    \end{theorem}
    \begin{theorem}
        \textbf{(Test for Divergence:)} This is the contrapositive of the previous theorem. If $a_k \not\to 0$ as $k\to \infty$, then $\sum_{k=0}^\infty a_k$ diverges.
    \end{theorem}
\end{itemize}