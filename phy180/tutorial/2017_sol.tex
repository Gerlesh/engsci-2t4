\documentclass{article}
\usepackage{phy}

\title{PHY180: Classical Mechanics \\ 2017 Exam Solutions}
\author{QiLin Xue}
\date{\today}
\usepackage{lmodern}
\usepackage{tikz}
\usepackage{pgfplots}
\usepackage{bm}
\usepackage{textcomp}
\usepackage{graphicx}
\usepackage{bbm}
\usepackage{parskip}
\usepackage{calc}
\usepackage{tkz-euclide}
\usetkzobj{all}
\usepackage{multicol}
\usepackage[inline]{asymptote}
\usepackage{siunitx}
\usetikzlibrary{arrows}
\usetikzlibrary{calc}

\pgfplotsset{compat=1.16}
\setlength\parindent{0pt}
\everymath{\displaystyle}
\usepackage{makeidx}
\makeindex



\let\oldtextbf\textbf
\renewcommand{\textbf}[1]{\oldtextbf{#1}\index{#1}}

\begin{document}

\maketitle
\section*{Problem 1}
\textbf{(a)} We integrate with respect to time to get velocity:
\begin{equation}
    v(t) = \int 5-t \dd{t} = 5t -\frac{1}{2}t^2 - 8
    \label{eq:}
\end{equation}
where I have added in the condition $v(0)=-8\si{\meter\per\second}$. This is zero when:
\begin{equation}
    t^2-10t+16=0 \implies (t-8)(t-2)=0
    \label{eq:}
\end{equation}
So the particle will stop at $t=2\si{\second}$ and $t=8\si{\second}$.

\textbf{(b)} We integrate again:
\begin{equation}
    x(t) = \int (5t-\frac{1}{2}t^2-8) \dd{t} = \frac{5}{2}t^2-\frac{1}{6}t^3-8t+10
    \label{eq:}
\end{equation}
using the condition that $x(0)=10$. At $t=3\si{\second}$, the position of the particle is:
\begin{equation}
    x(3) =4 \si{\meter}
    \label{eq:}
\end{equation}
\textbf{(c)} A critical point for the velocity of the particle occurs when $v'(t)=a(t)=0$, or when $t=5\si{\second}$, and the particle is travelling at a speed of:
\begin{equation}
    v(4) = 4\si{\meter\per\second}
    \label{eq:}
\end{equation}
Note that the initial and end points do not satisfy this as they are both moving in the $-\hat{i}$ direction.

\section*{Problem 2}
\textbf{(a)} Before it hits object $m$, it is moving at a speed of $v=\sqrt{2gh}$, from conservation of energy.

\textbf{(b)} We have a perfectly inelastic collision, where momentum is conserved:
\begin{equation}
    M\sqrt{2gh} = (M+m)v_A \implies v_A = \frac{M}{M+m}\sqrt{2gh}
    \label{eq:}
\end{equation}
\textbf{(c)} For the minimum speed, they are just about to leave contact with the track so the normal force becomes zero at the top of the track. Therefore:
\begin{equation}
    (m+M)g = \frac{(m+M)v_A'^2}{R}
    \label{eq:}
\end{equation}
where from conservation of energy, $v_A'$ is the speed at the top of the track and is given by:
\begin{equation}
    \frac{1}{2}(M+m)v_A^2 = \frac{1}{2}(M+m)v_A'^2 + (m+M)g(2R) \implies v_A' = \sqrt{v_A^2-4gR}
    \label{eq:}
\end{equation}
Substituting this into the force balance equation gives:
\begin{equation}
    gR = v_A^2-4gR \implies v_A = \sqrt{5gR}
    \label{eq:}
\end{equation}

\section*{Problem 3}
\textbf{(a)} The distance the contact point between the applied force and the rod moves is given by:
\begin{equation}
    y\tan(\Delta\theta)
    \label{eq:}
\end{equation}
so the work done is:
\begin{equation}
    W = Fy\tan(\Delta\theta) = 3.5\si{\joule}
    \label{eq:}
\end{equation}
\textbf{(b)} Let $y=0$ to be at the top of the rod. From conservation of energy, we have:
\begin{equation}
    -Mg\frac{L}{2} + W_\text{ext} = -Mg\frac{L}{2}\cos\theta  + K
    \label{eq:}
\end{equation}
Solving for $K$ gives:
\begin{equation}
    K = \frac{MgL}{2}\left(\cos\Delta\theta-1\right) + W_\text{ext} = 1.4\si{\joule}
    \label{eq:}
\end{equation}

\section*{Problem 4}
Consider a force balance on the mass:
\begin{equation}
    m_2(a-\alpha R) = T - m_2g
    \label{eq:}
\end{equation}
Note that the $a$ comes from the acceleration of the elevator and the $-\alpha R$ comes from the acceleration of the block with respect to the spool. A torque balance gives:
\begin{equation}
    I_\text{cm}\alpha = TR = m_2(g+a)R-m_2\alpha R^2
    \label{eq:}
\end{equation}
Solving for $\alpha$ then gives:
\begin{equation}
    \alpha = \frac{m_2R(a+g)}{I_\text{cm}+m_2R^2}
    \label{eq:}
\end{equation}
\section*{Problem 5}
Some orbital mechanics is needed for this problem, so feel free to skip this.

\textbf{(a)} We can calculate the center of mass in the radial direction away from mass $M$ to be:
\begin{equation}
    x_\text{cm} = \frac{2MD}{M+2M} = \frac{2}{3}D
    \label{eq:}
\end{equation}
Therefore, a force balance on mass $M$ gives:
\begin{equation}
    M\omega^2\left(\frac{2}{3}D\right) = \frac{GM(2M)}{D^2} \implies \omega = \sqrt{\frac{3GM}{D^3}}
    \label{eq:}
\end{equation}
\textbf{(b)} The angular velocity of both masses are the same about their centroid. Their respective velocities are $v=\frac{2}{3}\omega D$ and $v=\frac{1}{3}\omega D$, so their total kinetic energy is:
\begin{equation}
    K = \frac{1}{2}(M)\frac{4D}{9}\frac{3GM}{D^3} + \frac{1}{2}(2M)\frac{D}{9}\frac{3GM}{D^3} = \frac{GM^2}{D^2}
    \label{eq:}
\end{equation}
\textbf{(c)} The potential energy is:
\begin{equation}
    U = -\frac{GM(2M)}{D^2} = -\frac{2GM^2}{D^2}
    \label{eq:}
\end{equation}
Note that $2K+U=0$. This is a special case of a well known result called the \textbf{Virial Theroem}.
\section*{Problem 6} 
Consider a tiny displacement $\theta$ of the rod. The spring would exert a torque of:
\begin{equation}
    \tau_s = -k\Delta x \frac{L\cos\theta}{2}
\end{equation}
Approximating $\cos\theta\approx 1$ and $\Delta x = \frac{L}{2}\theta$, we get the torque from the spring to be:
\begin{equation}
    \tau_s = -\frac{kL^2}{4}\theta 
    \label{eq:}
\end{equation}
and the torque from gravity is:
\begin{equation}
    \tau_g = -mg\frac{L}{2}\sin\theta \approx -\frac{mgL}{2}\theta 
    \label{eq:}
\end{equation}
such that the torque balance equation becomes:
\begin{equation}
    \frac{1}{12}mL^2\alpha = -\left(\frac{mgL}{2} + \frac{kL^2}{4}\right)\theta \implies \alpha = -\left(\frac{6g}{L}+\frac{3k}{m}\right)\theta 
    \label{eq:}
\end{equation}
so the angular frequency is:
\begin{equation}
    \omega = \sqrt{\frac{6g}{L}+\frac{3k}{m}}
    \label{eq:}
\end{equation}

\end{document}
