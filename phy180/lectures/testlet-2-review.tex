\documentclass{article}
\usepackage{phy}

\title{PHY180: Testlet 2 Review}
\author{QiLin Xue}
\date{Fall 2020}
\usepackage{lmodern}
\usepackage{tikz}
\usepackage{pgfplots}
\usepackage{bm}
\usepackage{textcomp}
\usepackage{graphicx}
\usepackage{bbm}
\usepackage{parskip}
\usepackage{multicol}
\usetikzlibrary{arrows}
\pgfplotsset{compat=1.16}
\setlength\parindent{0pt}
\everymath{\displaystyle}
\usepackage{makeidx}
\makeindex
\let\oldtextbf\textbf
\renewcommand{\textbf}[1]{\oldtextbf{#1}\index{#1}}

\begin{document}

\maketitle
% \tableofcontents
\section{General Collisions}
There are three types of collisions:
\begin{itemize}
    \item \textbf{Elastic:} Energy is conserved.
    \item \textbf{Inelastic:} Some energy is lost.
    \item \textbf{Completely Inelastic:} Maximum loss of energy. Objects stick together afterwards.
\end{itemize}
The \textbf{coefficient of restitution} is defined as:
\begin{equation}
    e = \frac{v_{2,f}-v_{1,f}}{v_{1,i}-v_{2,i}}
    \label{eq:}
\end{equation}
For an elastic collision, $e=1$. The \textbf{conservation of momentum} applies when the net external force is zero:
\begin{equation}
    m_1v_{1,i}+m_2v_{2,i}=m_1v_{1,f}+m_2v_{2,f}.
    \label{eq:}
\end{equation}
The conservation of energy applies when no energy is being transferred in or out of the system (e.g. friction / explosions could make it lose energy):
\begin{equation}
    \frac{1}{2}m_1v_{1,i}^2+\frac{1}{2}m_2v_{2,i}^2=\frac{1}{2}m_1v_{1,f}^2+\frac{1}{2}m_2v_{2,f}^2.
\end{equation}
Oftentimes, problems are easier to deal with in the \textbf{center of mass frame}, as the net momentum is zero. The velocity of the center of mass is:
\begin{equation}
    v_\text{cm} = \frac{m_1v_1+m_2v_2}{m_1+m_2}
    \label{eq:}
\end{equation}
We can write the energy of the system as:
\begin{equation}
    K_\text{tot}=K_\text{cm}+K_\text{conv}
    \label{eq:}
\end{equation}
where:
\begin{equation}
    K_\text{cm}=\frac{1}{2}(M+m)v_\text{cm}^2
    \label{eq:}
\end{equation}
For a totally inelastic collision, the energy dissipated (which so happens to be the max) is given by the \textbf{convertible kinetic energy}:
\begin{equation}
    K_\text{conv}=\frac{1}{2}\mu v_{12}^2
    \label{eq:}
\end{equation}
where $\mu=\frac{m_1m_2}{m_1+m_2}$ is the \textbf{reduced mass} and $v_{12}$ is the relative velocity of the two objects.
\subsection{Tips and Tricks}
When solving a general collisions problem, there are two ways (that are equally as valid) to do so.
\begin{enumerate}
    \item Use conservation of momentum and the coefficient of restitution equation. Avoid using conservation of energy unless absolutely needed, since that will give a quadratic.
    \item View things in the center of mass frame where the total momentum is zero. In this frame, the objects simply collide, switch directions, and have their speeds scaled by a factor of $e$. Switching back to the lab frame gives us the answer.
\end{enumerate}
\begin{example}
    As an example and for practice, I will solve the most general problem. A mass $m_1$ has velocity $v_i$ and collides with a mass $m_2$ with velocity $u_i$. If the coefficient of restitution is $e$, what are their final velocities $v_f$ and $u_f$?
    \vspace{2mm}

    We have two equations:
    \begin{align}
        u_f-v_f&=ev_i-eu_i \\
        m_1v_i+m_2u_i &= m_1v_f+m_2u_f
    \end{align}
    Making the substitution $u_f=ev_i-eu_i+v_f$ gives us:
    \begin{align}
        m_1v_i+m_2u_i &= m_1v_f+m_2(ev_i-eu_i+v_f) \\
        m_1v_i+m_2u_i &= (m_1+m_2)v_f+m_2ev_i-m_2eu_i \\ 
        v_f &= \frac{m_1v_i+m_2u_i-em_2(u_i-v_i)}{m_1+m_2}
    \end{align}
    and similarly for $u_f$, we have:
    \begin{equation}
        u_f = \frac{m_1v_i+m_2u_i-em_1(v_i-u_i)}{m_1+m_2}
    \end{equation}
\end{example}
\begin{example}
\vspace{2mm}
We can also solve this from the center of mass frame. The CoM has a velocity of $v_\text{cm}=\frac{m_1v_i+m_2u_i}{m_1+m_2}$, so in this frame, the two objects have a velocity of $v'_i=v_i-v_\text{cm}$ and $u'_i=u_i-v_\text{cm}$. When they collide, their velocities change by a factor of $-e$ to become:
\begin{align}
    v'_f &= -e(v_i-v_\text{cm}) \\
    u'_f &= -e(u_i-v_\text{cm}) 
\end{align}
and shifting back to the lab frame, they have velocities of:
\begin{align}
    v'_f &= -e(v_i-v_\text{cm})+v_\text{cm}=\frac{m_1v_i+m_2u_i}{m_1+m_2}(1+e)-ev_i \\
    u'_f &= -e(u_i-v_\text{cm})+v_\text{cm}=\frac{m_1v_i+m_2u_i}{m_1+m_2}(1+e)-eu_i
\end{align}
which you can verify is the same result as in example 1.
\end{example}
\section{Energy}
For a closed system, the change in \textbf{mechanical energy} is zero:
\begin{equation}
    \Delta E_\text{mech}=\Delta K + \Delta U = 0
    \label{eq:}
\end{equation}
The potential energy near Earth's surface is:
\begin{equation}
    U_g = mgh
    \label{eq:}
\end{equation}

\end{document}