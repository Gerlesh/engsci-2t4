\documentclass{article}
\usepackage{phy}

\title{PHY180: Testlet 2 Unofficial Solutions}
\author{QiLin Xue}
\date{Fall 2020}
\usepackage{lmodern}
\usepackage{tikz}
\usepackage{pgfplots}
\usepackage{bm}
\usepackage{textcomp}
\usepackage{graphicx}
\usepackage{bbm}
\usepackage{parskip}
\usepackage{multicol}
\usetikzlibrary{arrows}
\pgfplotsset{compat=1.16}
\setlength\parindent{0pt}
\everymath{\displaystyle}
\usepackage{makeidx}
\makeindex
\let\oldtextbf\textbf
\renewcommand{\textbf}[1]{\oldtextbf{#1}\index{#1}}

\begin{document}

\maketitle
% \tableofcontents
\section*{MC Problem 1: Five Choices}
\begin{enumerate}
    \item Incorrect. Gravity is doing external work on the ball, so on the way up, the kinetic energy is constantly decreasing.
    \item Incorrect. The momentum of the ball is constantly changing since there is an external force (gravity) acting on the ball.
    \item Correct. Energy of the ball-Earth system is conserved and there is no external work (ignore air resistance).
    \item Correct. Gravity is now an internal force and the momentum of the ball-Earth system is conserved.
    \item Incorrect. Since $U_g$ changes, the kinetic energy of the ball-earth system also changes.
\end{enumerate}
\section*{MC Problem 2: Center of Mass}
False. We can find a counterexample where the center of mass does not evenly divide two sides with equal masses. Consider an object with mass $m=1\text{ kg}$ at $x=0$ and a mass $m=2\text{ kg}$ at $x=1 \text{ m}$. It's evident that the center of mass will lie somewhere in between these two point objects (specifically at $x=2/3\text{ m}$), where the mass on both sides is not the same.

\section*{FRQ Problem 1: Potential Energy}
From conservation of energy:
\begin{equation}
    U_i+K_i+K_f=U_f
    \label{eq:}
\end{equation}
We are told that the initial conditions: The object starts off at $x=1\text{ m}$ with a speed $v_0$. We wish to find the final speed $v_f$ at $x=x_f$. We are given the potential energy function to be: $U(x)=10/x^2$ and the mass of the object as $m$. Thus:
\begin{align*}
    10+\frac{1}{2}mv_0^2&=\frac{1}{2}mv_f^2+\frac{10}{x_f^2} \\
    v_f &= \sqrt{\frac{2}{m}\left(10+\frac{1}{2}mv_0^2-\frac{10}{x_f^2}\right)}
\end{align*}

\section*{FRQ Problem 2: Two colliding balls}
Hmm, not actually sure about this one and had to read up in Mazur what source energy was referring to. We shall assume that an additional kinetic energy of $\Delta E$ was added to the system. Let the two masses be $m$ and $M$ where $M$ is the mass originally at rest. If $m$ has an initial speed of $v$ and $M$ has an initial speed of $u=0$, then conservation of momentum + energy gives:
\begin{align*}
    mv &= mv_f+Mu_f \\
    \frac{1}{2}mv^2+\Delta E &= \frac{1}{2}mv_f^2+\frac{1}{2}Mu_f^2
\end{align*}
We have two unknowns and two equations, so we can solve for the final two velocities to get the relative velocity. There's probably a slicker way using the CoM but I panicked and defaulted to the bashy method.

\section*{FRQ Problem 3: Pulley}
Conservation of energy gives (here $m_2$ refers to the heavier object):
\begin{equation}
    K_i+U_i = K_f + U_f \implies K_f = -\Delta U 
    \label{eq:}
\end{equation}
where the change in potential energy is:
\begin{equation}
    \Delta U = m_1g \Delta h_1-m_2\Delta h_1 = (m_1-m_2)g\Delta h_1
    \label{eq:}
\end{equation}
so the final speed of both objects are:
\begin{equation}
    \frac{1}{2}(m_1+m_2)v^2 = (m_2-m_2)g\Delta h_1 \implies v = \sqrt{\frac{2(m_2-m_1)}{m_2+m_1}g\Delta h_1}
    \label{eq:}
\end{equation}
and the velocity of the center of mass is given by:
\begin{equation}
    v_\text{cm} = \frac{m_1v+m_2(-v)}{m_1+m_2} 
    \label{eq:}
\end{equation}
taking upwards to be positive. This should give a negative quantity.

\section*{FRQ Problem 4: Coefficient of Restitution}
Let the height that the first pendulum decreases to be $f$ (e.g. $f=0.101$ for $10.1\%$.) The initial speed of the first mass is $\sqrt{2gh}$ if the initial height is $h$. This is a result from conservation of energy. Similarly the final speed right after the collision of the first mass is $\sqrt{2gfh}$. Conservation of momentum then gives:
\begin{equation}
    m\sqrt{2gh}+m(0)=m\sqrt{2gfh}+mu \implies u =\sqrt{2gh}\left(1-\sqrt{f}\right)
    \label{eq:}
\end{equation}
where $u$ is the unknown speed of the second mass. The coefficient of restitution is given by:
\begin{equation}
    e = \frac{u-\sqrt{2gfh}}{\sqrt{2gh}-0} = \frac{\sqrt{2gh}\left(1-\sqrt{f}\right)-\sqrt{2gfh}}{\sqrt{2gh}}
    \label{eq:}
\end{equation}
we can divide top and bottom by $\sqrt{2gh}$ to get:
\begin{equation}
    e = \frac{1-\sqrt{f}-\sqrt{f}}{1} = 1-2\sqrt{f}
    \label{eq:}
\end{equation}
If you have $f=10.1\%$, this corresponds to a CoR of $e=0.364$. As a quick check, note that if $f=0$, then we would have $e=1$ which corresponds to an elastic collision. This makes sense since in a Newton's cradle, the incoming mass stays still afterwards.
\end{document}